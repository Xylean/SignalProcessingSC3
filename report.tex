\documentclass[french, a4paper, 12pt, openany]{book}

\usepackage[utf8]{inputenc}
\usepackage[T1]{fontenc}
\usepackage{babel}
\usepackage{amssymb}
\usepackage{amsmath}
\usepackage[top=2cm, bottom=2cm, left=2cm, right=2cm]{geometry}

\title{\Huge{Rapport Travaux Pratiques \\ SC3}}
\author{\\ Théo Bertrand \& Xyléan Broeders \\ Fi 2021}
\date{19 Décembre 2018}

%Glossary

\begin{document}

%Title of the document
\maketitle

\chapter{Analyse du signal}
	Durant ce TP, nous allons nous intéresser à l'étude d'un signal (ici le nX). Dans un premier temps, nous allons étudier les caractéristiques physiques du signal.  Pour ensuite, pouvoir déterminer si ce signal est stationnaire du second ordre. Finalement, nous allons essayer (dans la mesure du possible) de reconstituer le signal d'origine.
\section{Etude générale du signal}
	Nous observons tout d'abord que les valeurs de ce signal sont comprises entre XX et XX, et qu’il a une durée de XX secondes. Intéressons-nous maintenant à plusieurs de ses caractéristiques et notamment sa moyenne, sa variance, et son autocorrélation.
  \begin{figure}[ht]
    \begin{center}
     % \includegraphics[scale=0.5]{images/smthg.png}
    \end{center}
    \caption{Recapitulatif des valeurs obtenues}
    \label{Recapitulatif des valeurs obtenues}
  \end{figure}

\section{Stationnarité du 2\textsuperscript{nd} orde}
	Pour rappel, un signal est dit stationnaire du second ordre si ses propriétés statistiques ne varient pas en fonctions du temps. C'est-à-dire s'il vérifie :
  \begin{center}
  	\begin{math}\mathbb{E}[X(t)^2]<\infty\end{math} \\
  	\begin{math}\mathbb{E}[X(t)]=Constante\end{math} \\
  	\begin{math}\mathbb{E}[X(t)X(t)^*(t-\tau)]=C_X(\tau)\end{math}
  \end{center}
	On constate que le signal n'est pas stationnaire du second ordre, en effet son autocorrélation dépendant du temps. Néanmoins, nous observons qu'elle est constante sur certains intervalles de temps. On peut ainsi dénombrer X "blocs" :

  \begin{description}
    \item[Bloc 1 : ]de XX à XX
    \item[Bloc 2 : ]de XX à XX
    \item[Bloc 3 : ]de XX à XX
    \item[Bloc 4 : ]de XX à XX
  \end{description}
  \subsection{Bloc 1}
	Les variations de l'autocorrélation peuvent être considérées comme négligeables, on peut donc avec ce postulat, considérer ce bloc comme stationnaire du second ordre.
  \subsection{Bloc 2}
  \subsection{Bloc 3}
  \subsection{Bloc 4}

\chapter{Lancement des fonctions}
\end{document}
