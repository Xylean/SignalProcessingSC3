\documentclass[french, a4paper, 12pt, openany]{book}

\usepackage[utf8]{inputenc}
\usepackage[T1]{fontenc}
\usepackage{babel}
\usepackage{amssymb}
\usepackage{amsmath}
\usepackage[top=2cm, bottom=2cm, left=2cm, right=2cm]{geometry}

\title{Rapport Travaux Pratiques \\ SC3}
\author{Théo Bertrant \& Xyléan Broeders \\ Fi 2021}
\date{19 Décembre 2018}

%Glossary
\makeglossaries

\begin{document}

%Title of the document
\maketitle
Durant ce TP Nous allons nous interéssés à l'étude d'un signal (ici le n°X). Nous allons dans un premier temps étudier ces caractéristiques physiques. Ensuite nous alons déduire des analyses précendentes, si oui ou non ce signal est stationnaire du 2\textsuperscript{nd} ordre. Enfin, nous allons essayer (dans la mesure du possible) de reconstituier le signal d'origine.
\section{Etude générale du signal}
  Nous observons tout d'abord que les valeurs de ce signal sont comprises entre XX et XX, et qu'il dure XX xx secondes.
  Interresons maintenant à plusieurs de ce caractéristiques, sa moyenne, variance, et son autocorrélation.
  \begin{figure}[ht]
    \begin{center}
      \includegraphics[scale=0.5]{images/smthg.png}
    \end{center}
    \caption{Recapitulatif des valeurs obtenues}
    \label{Recapitulatif des valeurs obtenues}
  \end{figure}

\section{Stationnarité du 2\textsuperscript{nd} orde}
  Pour rappel, un signal est dit staionnnaire du second ordre si s'est propriétés statistiques ne varient pas en fonctions du temps. C'est à dire s'il verifie :
  \begin{math}\mathbb{E}[X(t)^2]<\infty\end{math}
  \begin{math}\mathbb{E}[X(t)]=Constante<\infty\end{math}
  \begin{math}\mathbb{E}[X(t)X(t)^*(t-\tau)]=C_X(\tau)<\infty\end{math}
  \\
  On constate ainsi que le signal n'est pas stationnaire du 2\textsuperscript{nd} ordre, en effet sont autocorrélation est dépendante du temps. Néanmoins, nous observons qu'elle est conctate sur certains intervalles de temps. Nous définissons ainsi X "blocs" :
  \begin{description}
    \item[Bloc 1 : ]de XX à XX
    \item[Bloc 2 : ]de XX à XX
    \item[Bloc 3 : ]de XX à XX
    \item[Bloc 4 : ]de XX à XX
  \end{description}
  \subsection{Bloc 1}
  Les variations de l'autocorrélation pouvant être considérés comme négligeable, on considére ce bloc comme stationnaire du 2\textsuperscript{nd} ordre.
  \subsection{Bloc 2}
  \subsection{Bloc 3}
  \subsection{Bloc 4}


\end{document}
